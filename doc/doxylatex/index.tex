 \begin{DoxyVersion}{Version}
0.\+9.\+0 
\end{DoxyVersion}
\begin{DoxyAuthor}{Author}
Carsten Haitzler \href{mailto:raster@rasterman.com}{\tt raster@rasterman.\+com} 
\end{DoxyAuthor}
\begin{DoxyDate}{Date}
2000-\/2003
\end{DoxyDate}
\hypertarget{index_intro}{}\section{What is Epeg?}\label{index_intro}
An I\+M\+M\+E\+N\+S\+E\+L\+Y F\+A\+S\+T J\+P\+E\+G thumbnailer library A\+P\+I.

Why write this? It is a convenience library A\+P\+I to using libjpeg to load J\+P\+E\+G images destined to be turned into thumbnails of the original, saving information with these thumbnails, retreiving it and managing to load the image ready for scaling with the minimum of fuss and C\+P\+U overhead.

This means that it is insanely fast at loading large J\+P\+E\+G images and scaling them down to tiny thumbnails. The speedup that it provides will be proportional to the size difference between the source image and the output thumbnail size as a count of their pixels.

It makes use of libjpeg features of being able to load an image by only decoding the D\+C\+T coefficients needed to reconstruct an image of the size desired. This gives a massive speedup. If you do N\+O\+T try to access the pixels in a format other than Y\+U\+V (or G\+R\+A\+Y8 if the source is grascale), then it also avoids colorspace conversions as well.

Using the library is very easy; look at this example\+:


\begin{DoxyCode}
\textcolor{preprocessor}{#include <stdio.h>} \textcolor{comment}{/* for printf() */}
\textcolor{preprocessor}{#include <stdlib.h>} \textcolor{comment}{/* for exit() */}
\textcolor{preprocessor}{#include "Epeg.h"}

\textcolor{keywordtype}{int} main(\textcolor{keywordtype}{int} argc, \textcolor{keywordtype}{char} **argv)
\{
   Epeg\_Image *im;

   \textcolor{keywordflow}{if} (argc != 3) \{
        printf(\textcolor{stringliteral}{"Usage: %s input.jpg thumb.jpg\(\backslash\)n"}, argv[0]);
        exit(0);
   \}
   im = epeg_file_open(argv[1]);
   \textcolor{keywordflow}{if} (!im) \{
        printf(\textcolor{stringliteral}{"Cannot open %s\(\backslash\)n"}, argv[1]);
        exit(-1);
   \}
   
   epeg_decode_size_set(im, 128, 96);
   epeg_quality_set(im, 75);
   epeg_thumbnail_comments_enable(im, 1);
   epeg_file_output_set(im, argv[2]);
   epeg_encode(im);
   epeg_close(im);
   
   \textcolor{keywordflow}{return} 0;
\}
\end{DoxyCode}


This program exists as epeg\+\_\+test.\+c so that you can compile this program with as small a line as\+:

\begin{DoxyVerb}gcc epeg_test.c -o epeg_test `epeg-config --cflags --libs`
\end{DoxyVerb}


It is a very simple library that just makes life easier when trying to generate lots of thumbnails for large J\+P\+E\+G images as quickly as possible. Your milage may vary, but it should save you lots of time and effort in using libjpeg in general.

\begin{DoxyRefDesc}{Todo}
\item[\hyperlink{todo__todo000001}{Todo}]Check all input parameters for sanity. 

Actually report error conditions properly.\end{DoxyRefDesc}
